\chapter{\abstractname}

To understand the internal workings of brain neuroscientists have been trying to reconstruct its 'neural circuitry', also termed as a 'connectome' which illustrates how neurons are connected with each other. 

A connectome can be constructed by segmenting neurons in a 3D electron microscopy(EM) image and then by finding their inter-connections. Past decade has seen increasing use of Deep Learning based methods to improve segmentation of such EM images and it has been quite indispensable in improving the state of the art in EM segmentation. 

But the neuron tracing problem is a hard problem to solve in itself. This can be attributed to its large volume size, multiple image artifacts, numerous closely intertwined segments of vivid shapes etc. \textcolor{red}{show images illustrating problems}. Hence, the results of most algorithms are either not satisfactory or they are too slow and entail complicated postprocessing steps, limiting their scope to only inside a computer science lab, and far away from being used as a general tool directly by neuroscientists for new images.

This work tries to develop a new idea of building connectomes, bypassing the segmentation step. It builds upon from existing methods and approaches in EM segmentation but also leverages the tools and tricks of Machine Learning methods from natural image domain. 

The first part discusses the relevant works from the Connectomics area, their advantages and pitfalls. Apart from that, it also delves into skeletonization methods used for natural images, which inspires the
proposed method.







