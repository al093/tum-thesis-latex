% !TeX root = ../main.tex
% Add the above to each chapter to make compiling the PDF easier in some editors.

\chapter{Future Work}\label{chapter:further_work}
Our proposed method shows promising results for skeleton prediction in complicated 3D datasets but it still has unresolved false splits because the Tracking Network fails to correctly track in longer gaps. Also, our splitting method creates more than necessary splits which could be avoided. So further work would be to:
\begin{itemize}
	\item Improve splitting process by selectively splitting at false splits. Currently all junction points are split but it is not necessary as some of them are branch points in the same object. A classifier can be designed to discriminate between false merges and branch points.
	
	\item Improve the training ground truth for Tracking Network. If there is a large mismatch between ground truth skeletons and predicted over-split skeletons the Tracking network may not learn meaningful filters. To alleviate this, synthetic Tracking Network ground truth can be constructed.
	
\end{itemize}

Also, one could find better error metrics for evaluating instance skeletonization performance since binary pixel level F1 score does not care about instances and even instance F1 scores are biased towards split skeletons. One potential candidate would be estimated run length metric from Januszewski \etall~\cite{Januszewski2017FFN, Januszewski2018FFN}.
Last, as an application for skeletons, skeleton-assisted segmentation methods can be devised which could alleviate false merges and splits in segmentation methods and also use cheaper skeleton labels for training.
